
\documentclass[final]{beamer}

\usepackage[scale=1.24]{beamerposter} % Use the beamerposter package for laying out the poster
\usepackage{ amssymb }
%\usetheme{confposter} % Use the confposter theme supplied with this template
\usetheme[faculty=chemo]{fibeamer} % Uncomment to use Masaryk University's fibeamer theme instead.


\newlength{\sepwid}
\newlength{\onecolwid}
\newlength{\twocolwid}
\newlength{\threecolwid}
\setlength{\paperwidth}{46.8in} % A0 width: 46.8in
\setlength{\paperheight}{33.1in} % A0 height: 33.1in
\setlength{\sepwid}{0.024\paperwidth} % Separation width (white space) between columns
\setlength{\onecolwid}{0.21\paperwidth} % Width of one column
\setlength{\twocolwid}{0.451\paperwidth} % Width of two columns
\setlength{\threecolwid}{0.678\paperwidth} % Width of three columns
%-----------------------------------------------------------

\newcommand{\argmax}[1]{\underset{#1}{\operatorname{arg}\,\operatorname{max}}\;}

\usepackage{graphicx}  % Required for including images

\usepackage{booktabs} % Top and bottom rules for tables

%----------------------------------------------------------------------------------------
%	TITLE SECTION 
%----------------------------------------------------------------------------------------

\title{An Expert System for Medical Differential Diagnosis} % Poster title

\author{Mert Cem Ta\c{s}demir\\Advisor: Assoc. Prof. Ali Taylan Cemgil } % Author(s)


\institute{Bo\u{g}azi\c{c}i University Computer Engineering Department} % Institution(s)



\begin{document}

\addtobeamertemplate{block end}{}{\vspace*{1.3ex}} % White space under blocks
\addtobeamertemplate{block example end}{}{\vspace*{1.3ex}} % White space under example blocks
\addtobeamertemplate{block alerted end}{}{\vspace*{1.3ex}} % White space under highlighted (alert) blocks

\setlength{\belowcaptionskip}{1.3ex} % White space under figures
\setlength\belowdisplayshortskip{1.3ex} % White space under equations

\begin{frame} % The whole poster is enclosed in one beamer frame

%==========================Begin Head===============================
  \begin{columns}
 
  \begin{column}{0.15\linewidth}
  \centering
   \includegraphics[width=0.6\textwidth]{img/f2.eps}
    \hspace{5cm}
   \end{column}
   
   \begin{column}{0.7\linewidth}
    \vskip1cm
    \centering
    \usebeamercolor{title in headline}{\color{fg}\Huge{\textbf{\inserttitle}}\\[0.5ex]}
    \usebeamercolor{author in headline}{\color{fg}\Large{\insertauthor}\\[1ex]}
    \usebeamercolor{institute in headline}{\color{fg}\large{\insertinstitute}\\[1ex]}
    \vskip1cm
   \end{column}
   \vspace{1cm}
   \begin{column}{0.15\linewidth}
    \hspace{5cm}
   \centering
   \includegraphics[width=0.6\textwidth]{img/f2.eps}
   \end{column}
  \end{columns}

 \vspace{1cm}

%==========================End Head===============================

\begin{columns}[t] % The whole poster consists of three major columns, the second of which is split into two columns twice - the [t] option aligns each column's content to the top

\begin{column}{\sepwid}\end{column} % Empty spacer column

\begin{column}{\onecolwid} % The first column

%----------------------------------------------------------------------------------------
%	OBJECTIVES
%----------------------------------------------------------------------------------------

\begin{exampleblock}{Objectives}
 The project consists of two essential parts. One of them is to find the best set of diagnosis given a set of symptoms. The other is to find the best question to be asked to the patient to find the diagnosis.

\end{exampleblock}

%----------------------------------------------------------------------------------------
%	INTRODUCTION
%----------------------------------------------------------------------------------------


\begin{exampleblock}{Network Model \& Approach}

For the first part we tried to estimate maximum the probability of observing a disease combination  $d^*$ , given set of symptoms $s_{1:k}$ where $k<j$, and $d^*$ consists of $d_{1:m<t}$'s where $t$ is a constant that reflects the maximum number of concurrent diseases \textit{and we took it $3$ for the sake of computational complexity}. In other words, we tried to make a \textit{Maximum a-Posteriori(MAP)} estimation. \\
MAP: 
\begin{center}
$d^* = \argmax{d} p(s|d)p(d)$
\end{center}


%\vspace*{50px}
 \begin{figure}
\includegraphics[width=0.9\linewidth]{img/model.png}
\caption{Two layer Bayesian Network for QMR-DT}
\end{figure}

%This statement requires citation \cite{Smith:2012qr}.

\end{exampleblock}





%----------------------------------------------------------------------------------------

\end{column} % End of the first column

\begin{column}{\sepwid}\end{column} % Empty spacer column

\begin{column}{\twocolwid} % Begin a column which is two columns wide (column 2)

\begin{columns}[t,totalwidth=\twocolwid] % Split up the two columns wide column

\begin{column}{\onecolwid}\vspace{-.74in} % The first column within column 2 (column 2.1)


\begin{exampleblock}{State-of-Art}
\begin{itemize}
\item Old school knowledge-based systems are outdated
\item Neural Networks are on one cutting edge 
\item Bayesian methods are also quite popular. 
\item QMR-DT is the most popular probabilistic model on which this project is built.
\end{itemize}


\end{exampleblock}



\end{column} % End of column 2.1
\begin{column}{\sepwid}\end{column} % Empty spacer column

\begin{column}{\onecolwid}\vspace{-.74in} % The second column within column 2 (column 2.2)



\begin{exampleblock}{Question Asking Strategies}


\begin{itemize}
\item \textbf{Relative-Entropy Based Strategy :} Minimize the Shannon Entropy
\item \textbf{Strategy Based on Symptoms :} Rank the symptoms according to their number of occurance in distinct diseases
\item \textbf{Strategy Based on Diseases :} Rank the diseases according to the number of symtoms that they are related
\item \textbf{Strategy Based on Symptoms and RE :} Hybrid of first two
\end{itemize}

\end{exampleblock}



\end{column} % End of column 2.2

\end{columns} % End of the split of column 2 - any content after this will now take up 2 columns width



\begin{alertblock}{Important Result}

Large networks require so much power that makes the model nonutilizable after some point. So the model can be used for reduced problems.

\end{alertblock} 

%----------------------------------------------------------------------------------------

\begin{columns}[t,totalwidth=\twocolwid] % Split up the two columns wide column again

\begin{column}{\onecolwid} % The first column within column 2 (column 2.1)

%----------------------------------------------------------------------------------------
%	MATHEMATICAL SECTION
%----------------------------------------------------------------------------------------

\begin{exampleblock}{Mathematical Section}
Data description:
\begin{equation}
ds_{ij}\sim \mathcal{BE}(\big[0,1\big]; \pi_{0}, 1-\pi_{0}), \pi_{0}>>1-\pi_{0}
\end{equation}

Probability of not observing $s_{j}$ is:
\begin{equation*}
p(s_{i}=0|d) = \theta_{0}\prod_{j}\theta^{ds_{ij}d_{j}}
\end{equation*}
Probability of observing $s_{j}$ is:
\begin{equation*}
p(s_{i}=1|d) = 1- \theta_{0}\prod_{j}\theta^{ds_{ij}d_{j}}
\end{equation*}
$\theta_{0}$ is the probability of not observing a symptom when no disease related to it is present.
\\
%Quickscore decreases the multiplication of the previous two to:
%\begin{equation*}
%p(s^{-}) = \prod_{j}\Big[p(s^{-}|only d_{i})p(d_{i}^{+})+p(d_{i}^{-}))
%\end{equation*}

Minus log posterior: 
\begin{equation*}
-\mathcal{L}=-logp(d|s) \propto -logp(s|d)+log(1/p(d))
\label{eqn:Minus log of posterior}
\end{equation*}





\end{exampleblock}

%----------------------------------------------------------------------------------------

\end{column} % End of column 2.1
\begin{column}{\sepwid}\end{column} % Empty spacer column

\begin{column}{\onecolwid} % The second column within column 2 (column 2.2)

%----------------------------------------------------------------------------------------
%	RESULTS
%----------------------------------------------------------------------------------------

\begin{exampleblock}{Results}

\begin{figure}
\includegraphics[width=0.8\linewidth]{img/SmallNetwork10_20.png}
\caption{QGS Comparison - Small Network(20 Symptoms)}
\end{figure}

\vspace{2cm}
\begin{figure}
\includegraphics[width=0.8\linewidth]{img/LargeNetwork30_100.png}
\caption{QGS Comparison - Large Network(100 Symptoms)}
\end{figure}

\end{exampleblock}

%----------------------------------------------------------------------------------------

\end{column} % End of column 2.2

\end{columns} % End of the split of column 2

\end{column} % End of the second column

\begin{column}{\sepwid}\end{column} % Empty spacer column

\begin{column}{\onecolwid} % The third column



\begin{exampleblock}{Conclusion}
As mentioned, it's not working efficiently or even not tolerable for large networks. So, nonetheless it's very efficient and high scoring model for small networks, a scalable model seems necessary.


\end{exampleblock}



\begin{exampleblock}{Future Work}

Delving into variational methods and their application to this domain since the inference is infeasible in large models. Besides them, Monte Carlo methods should be examined. \\
As Confucius said "The man who moves a mountain begins by carrying away small stones." and this project was one of the smallest ones.


\end{exampleblock}

\begin{block}{Special Thanks to}

\begin{itemize}
\item My advisor Ali Taylan Cemgil for allowing me to participate this presentation after too much confusion of which I am responsible for, and helping me to make the simple and extremely necessary move.
\item H{\i}d{\i}r Y\"{u}z\"{u}g\"{u}zel for his great project -UzmanDoktar, project report and implementation.
\end{itemize}

\end{block}

\begin{tabular}{rr}
\hspace{0.02\linewidth} & \includegraphics[scale=2.4]{img/bayes.jpeg}
\end{tabular}

%----------------------------------------------------------------------------------------

\end{column} % End of the third column

\begin{column}{\sepwid}\end{column} % Empty spacer column

\end{columns} % End of all the columns in the poster

\end{frame} % End of the enclosing frame
%\end{darkframes} % Uncomment for dark theme
\end{document}

